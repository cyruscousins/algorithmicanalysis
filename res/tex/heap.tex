Perhaps the quintisential priority queue data structure, heaps are popular for their simplicity and speed.  Asymptotically superior data structures exist, but heaps are still widely used in practice due to their low overhead, small constant cost factors, and practicality.  Discussion of interesting exact costs follow.

\begin{enumerate}[a):]

\item \texttt{construct}: Using the linear time heap construction algorithm, analysis through convergent series yields the bound of $2^{\ceil{\log_2 n}}$.

\item $E[$\texttt{decrease\_key}$]$: This expectation holds when the position of the key to be decreased is assumed to be random and the value to be decreased to is minimal for the heap\footnote{This decision was made because a more complicated assumption, which could yield a lower bound, would involve making assumptions about the values in the heap that are likely to be violated frequently.} and $n$ is given by $2^x - 1$ for some $x \in \mathbb{Z}$.  For other $n$, this is still an upper bound.  Now, $E[$\texttt{decrease\_key}$] = \sum_{i = 0}^n \frac{1}{n} \ceil{\log_2 i} = \frac{1}{n} \sum_{i = 0}^n \ceil{\log_2 n}$.  
Now, by a counting argument, if we make the further restriction that $x \geq 3$ (to ensure the heap has at least 3 rows), the expected cost of the bottommost three rows of the heap sum to $\sum_{i = 2 ^ {\ceil{\log_2 n} - 2}}^n = \frac{1}{n - 2 ^ {\ceil{\log_2 n} - 2}} \sum_{i = 2 ^ {\ceil{\log_2 n} - 2}}^n \Big( \ceil{\log_2 n} - 1 \Big)$.  
The remainder of the elements in the heap have total cost $\frac{1}{2 ^ {\ceil{\log_2 n} - 2}} \sum_0^{2 ^ {\ceil{\log_2 n} - 2} - 1} y$, for some $y \leq \ceil{\log_2 n} - 1$.  Using these results, we can note that $E[$\texttt{decrease\_key}$]$ is bounded above by $\ceil{\log_2 n} - 1 = \ceil{\log_2 n} - 1$.

\end{enumerate}
